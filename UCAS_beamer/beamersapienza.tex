\documentclass{beamer}
\usepackage{amsfonts,amsmath,oldgerm}
\usetheme{sintef}
\usepackage{xeCJK}
\usepackage{tikz}
\usetikzlibrary{shapes,arrows,positioning,fit,backgrounds}

\newcommand{\testcolor}[1]{\colorbox{#1}{\textcolor{#1}{test}}~\texttt{#1}}

\usefonttheme[onlymath]{serif}

\titlebackground*{assets/background}

\newcommand{\hrefcol}[2]{\textcolor{cyan}{\href{#1}{#2}}}

\title{分布式键值存储系统}
% \course{Master's Degree in Computer Science}
\author{翟锦洋、付涛}
% \IDnumber{1234567}
\date{\today}

\begin{document}
\maketitle

\section{项目概述}

\begin{frame}{实现功能}
    \begin{itemize}
        \item 设计并实现键值数据的多机分布式数据存储,采用RPC通信协议。
        \item 支持数据的增加、修改和删除操作
        \item 支持记录的多副本存储
        \item 提供读写接口/检索界面访问数据存储
        \item 支持raft一致性协议模块,故障情况下实现保证可靠存储
        \item 实现元数据管理
    \end{itemize}
\end{frame}

\section{meta-server架构}

\footlinecolor{sintefyellow}
\begin{frame}{整体架构图}
    \vspace*{0cm}
    \begin{center}
        \includegraphics[width=\textwidth,height=0.9\textheight,keepaspectratio]{overall.png}
    \end{center}
\end{frame}

\begin{frame}[fragile]{整体架构}
    \begin{columns}
        \begin{column}{0.5\textwidth}
            \begin{block}{应用架构}
                应用采用三层分布式架构,客户端通过代理服务器与节点服务器通信。代码组织上,使用 Python 语言实现 meta-server;使用 XML-RPC 实现客户端与服务器之间的通信;使用 Flask 实现 HTTP 测试服务器,使用 requests 库实现节点服务器与 Raft 集群的 HTTP 通信。
            \end{block}
        \end{column}
        \begin{column}{0.5\textwidth}
            \fontsize{8.1pt}{9.72pt}\selectfont
            \begin{verbatim}
meta-server/
|-- client.py // 客户端实现
|-- proxy_server.py // 代理服务器
|-- node_server.py // 节点服务器
|-- test_flask.py // Flask测试服务器
|-- api.md // API文档
|-- requirements.txt // 依赖管理
            \end{verbatim}
        \end{column}
    \end{columns}
\end{frame}

\begin{frame}{系统架构概览}
    \begin{columns}
        \begin{column}{0.5\textwidth}
            \begin{block}{三层架构}
                \begin{itemize}
                    \item \textbf{客户端}:用户交互界面
                    \item \textbf{代理服务器}:请求路由和负载分配
                    \item \textbf{节点服务器}:业务逻辑处理
                \end{itemize}
            \end{block}
        \end{column}
        \begin{column}{0.5\textwidth}
            \begin{block}{通信方式}
                \begin{itemize}
                    \item 客户端 $\leftrightarrow$ 代理服务器:XML-RPC
                    \item 代理服务器 $\leftrightarrow$ 节点服务器:XML-RPC
                    \item 节点服务器 $\leftrightarrow$ Raft集群:HTTP
                \end{itemize}
            \end{block}
        \end{column}
    \end{columns}
\end{frame}

\begin{frame}{组件说明}
    \begin{block}{客户端 (client.py)}
        \begin{itemize}
            \item \textbf{功能}:用户交互界面
            \item \textbf{通信方式}:XML-RPC
            \item \textbf{主要特性}:
                  \begin{itemize}
                      \item 用户认证(用户名/密码)
                      \item 客户端ID分配
                      \item 命令解析与执行
                  \end{itemize}
        \end{itemize}
    \end{block}
\end{frame}

\begin{frame}{组件说明}
    \begin{block}{代理服务器 (proxy\_server.py)}
        \begin{itemize}
            \item \textbf{功能}:请求路由和负载分配
            \item \textbf{端口}:21000
            \item \textbf{主要特性}:
                  \begin{itemize}
                      \item 用户认证管理
                      \item 客户端ID分配与管理
                      \item 命令转发到对应的节点服务器
                      \item 响应格式化与错误处理
                  \end{itemize}
        \end{itemize}
    \end{block}
\end{frame}

\begin{frame}{组件说明}
    \begin{block}{节点服务器 (node\_server.py)}
        \begin{itemize}
            \item \textbf{功能}:业务逻辑处理层
            \item \textbf{端口}:20000 + server\_id
            \item \textbf{主要特性}:
                  \begin{itemize}
                      \item 键值对操作(PUT、GET、DEL、LIST)
                      \item 集群管理操作(ADD-LEARNER、CHANGE-MEMBERSHIP、METRICS)
                      \item 操作日志记录
                      \item HTTP请求封装
                  \end{itemize}
        \end{itemize}
    \end{block}
\end{frame}

\begin{frame}{数据操作API}
    \begin{block}{支持的命令}
        \begin{itemize}
            \item \textbf{PUT}:添加或更新键值对
            \item \textbf{GET}:读取单个键值
            \item \textbf{DEL}:删除键值对
            \item \textbf{LIST}:读取所有键值对
        \end{itemize}
    \end{block}
\end{frame}

\begin{frame}[fragile]{PUT操作}
    \begin{block}{功能特点}
        \begin{itemize}
            \item \textbf{API端点}:\texttt{POST /write}
            \item \textbf{请求格式}:\texttt{\{"Put": \{"key": "key", "value": "value"\}\}}
            \item \textbf{响应}:\texttt{"Ok"} 或 \texttt{"Err"}
            \item \textbf{特点}:自动区分添加和更新操作
        \end{itemize}
    \end{block}
    \begin{block}{代码示例}
        \small
        \begin{verbatim}
json_data = {"Put": {"key": key, "value": value}}
response = self._http_request('/write', json_data=json_data)
\end{verbatim}
    \end{block}
\end{frame}

\begin{frame}[fragile]{GET操作}
    \begin{block}{功能特点}
        \begin{itemize}
            \item \textbf{API端点}:\texttt{POST /read}
            \item \textbf{请求格式}:字符串 \texttt{"key"}
            \item \textbf{响应格式}:\texttt{\{"Ok": "value"\}} 或 \texttt{\{"Ok": ""\}}
            \item \textbf{特点}:支持缓存机制(可选)
        \end{itemize}
    \end{block}
    \begin{block}{代码示例}
        \small
        \begin{verbatim}
json_data = key
response = self._http_request('/read', json_data=json_data, method='POST')
\end{verbatim}
    \end{block}
\end{frame}

\begin{frame}{DEL和LIST操作}
    \begin{columns}
        \begin{column}{0.5\textwidth}
            \begin{block}{DEL操作}
                \begin{itemize}
                    \item \textbf{API端点}:\texttt{POST /write}
                    \item \textbf{请求格式}:\texttt{\{"Del": \{"key": "key"\}\}}
                    \item \textbf{响应}:\texttt{"Ok"} 或 \texttt{"Err"}
                \end{itemize}
            \end{block}
        \end{column}
        \begin{column}{0.5\textwidth}
            \begin{block}{LIST操作}
                \begin{itemize}
                    \item \textbf{API端点}:\texttt{GET /read-all}
                    \item \textbf{响应格式}:\texttt{\{"Ok": [\{...\}, ...]\}}
                    \item \textbf{特点}:返回完整数据库快照
                \end{itemize}
            \end{block}
        \end{column}
    \end{columns}
\end{frame}

\begin{frame}{集群管理API}
    \begin{block}{ADD-LEARNER}
        \begin{itemize}
            \item \textbf{功能}:添加Raft节点作为learner
            \item \textbf{API端点}:\texttt{POST /add-learner}
            \item \textbf{请求格式}:\texttt{[node\_id, "api\_addr"]}
        \end{itemize}
    \end{block}
    \begin{block}{CHANGE-MEMBERSHIP}
        \begin{itemize}
            \item \textbf{功能}:改变集群成员关系
            \item \textbf{API端点}:\texttt{POST /change-membership}
            \item \textbf{请求格式}:\texttt{[node\_id1, node\_id2, ...]}
        \end{itemize}
    \end{block}
\end{frame}

\begin{frame}{METRICS操作}
    \begin{block}{功能}
        查询集群状态,返回详细信息:
        \begin{itemize}
            \item 节点基本信息:ID、任期、状态、Leader
            \item 投票信息、日志信息、成员配置
            \item 心跳信息、复制信息
        \end{itemize}
    \end{block}
    \begin{block}{用途}
        集群监控和诊断
    \end{block}
\end{frame}

\begin{frame}[fragile]{客户端连接流程}
    \begin{block}{认证和连接}
        \begin{verbatim}
def connect(self, username, password):
    self.port = '21000'
    self.proxy = xmlrpclib.ServerProxy(
        'http://localhost:' + self.port)
    if self.proxy.authenticate(username, password):
        self.id = self.proxy.get_id()
        return self.id
    return None
\end{verbatim}
    \end{block}
\end{frame}

\begin{frame}[fragile]{命令处理流程}
    \begin{block}{代理服务器命令路由}
        \begin{verbatim}
def function(self, client_id, clause):
    clause = clause.lower().strip().split()
    command = clause[0]
    if command in ['put', 'get', 'del', ...]:
        method_name = command
        server_function = getattr(self, method_name)
        return server_function(client_id, clause)
\end{verbatim}
    \end{block}
\end{frame}

\begin{frame}[fragile]{PUT操作实现}
    \begin{block}{节点服务器PUT方法}
        \small
        \begin{verbatim}
def put(self, key, value):
    existing_value = self.get(key)
    json_data = {"Put": {"key": key, "value": value}}
    response = self._http_request('/write', json_data=json_data)
    if response == "Ok":
        self.write_log(f"添加key:{key},value:{value}")
        return True
    return False
\end{verbatim}
    \end{block}
\end{frame}


\begin{frame}{Flask API测试端点}
    \begin{block}{HTTP接口}
        \begin{itemize}
            \item \texttt{POST /write}:写入操作(Put/Del)
            \item \texttt{POST /read}:读取单个键值
            \item \texttt{GET /read-all}:读取所有键值
            \item \texttt{POST /add-learner}:添加learner节点
            \item \texttt{POST /change-membership}:改变成员关系
            \item \texttt{GET /metrics}:查询集群状态
            \item \texttt{GET /health}:健康检查
        \end{itemize}
    \end{block}
\end{frame}

\begin{frame}[fragile]{Flask API示例}
    \begin{block}{写入操作实现}
        \small
        \begin{verbatim}
@app.route('/write', methods=['POST'])
def write():
    data = request.get_json()
    if 'Put' in data:
        put_data = data['Put']
        key = put_data.get('key')
        value = put_data.get('value')
        with db_lock:
            database[key] = value
        return "Ok", 200
    ...
\end{verbatim}
    \end{block}
\end{frame}

\footlinecolor{}
\section{kv-store架构}

\footlinecolor{sintefgreen}
\begin{frame}[fragile]{整体架构}
    \begin{columns}
        \begin{column}{0.5\textwidth}
            \begin{block}{应用架构}
                应用采用多节点分布式架构,每个节点运行独立的 Raft 实例。代码组织上,使用 rust 语言实现 kv-store;使用 openraft 实现 raft 协议提供一致性保证;使用 actix-web 实现 http 服务器,使用 rust bTreeMap 和 RocksDB 实现存储层。
            \end{block}
        \end{column}
        \begin{column}{0.5\textwidth}
            \fontsize{8.1pt}{9.72pt}\selectfont
            \begin{verbatim}
kv-store/
|-- src/
|   |--  lib.rs // 集群实现
|   |-- app.rs // 集群声明
|   |-- bin/main.rs // 主入口
|   |-- store/ // 存储层
|   |   |-- mod.rs // 存储层实现
|   |    -- log_store.rs // 日志存储
|   |-- network // 网络层
|   |   |-- mod.rs // 网络层实现
|   |   |-- api.rs // 业务 API
|   |   |-- management.rs // 管理 API
|   |    -- raft.rs // raft协议实现
|   |-- openraft_network // openraft网络层
|-- test.sh // 测试脚本
 -- start.sh // 启动脚本
                \end{verbatim}
        \end{column}
    \end{columns}
\end{frame}

\begin{frame}[plain]
    \vspace*{0cm}
    \begin{center}
        \includegraphics[width=\textwidth,height=\textheight,keepaspectratio]{architecture.png}
    \end{center}
\end{frame}

\begin{frame}{核心组件}
    \begin{block}{1. Raft 节点 (\texttt{src/lib.rs})}
        \begin{itemize}
            \item 每个节点包含一个 \texttt{openraft::Raft<TypeConfig>} 实例
            \item 节点通过 HTTP 服务(Actix-web)暴露 RPC 和业务 API
        \end{itemize}
    \end{block}
    \begin{block}{2. 存储层 (\texttt{src/store/})}
        \begin{itemize}
            \item \textbf{日志存储} (\texttt{RocksLogStore}): 基于 RocksDB 持久化 Raft 日志条目
            \item \textbf{状态机存储} (\texttt{StateMachineStore}): 管理 KV 数据状态,实现日志应用到状态机
        \end{itemize}
    \end{block}
    \begin{block}{3. 网络层 (\texttt{src/openraft\_network/})}
        \begin{itemize}
            \item 实现 \texttt{RaftNetwork} 和 \texttt{RaftNetworkFactory},处理节点间 RPC 通信
            \item 通过 HTTP 传输 Raft 协议消息(vote、append\_entries、install\_snapshot)
        \end{itemize}
    \end{block}
\end{frame}

\begin{frame}{存储层实现}
    \begingroup
    \setbeamerfont{frametitle}{size=\fontsize{8pt}{10pt}\selectfont}
    \setbeamerfont{normal text}{size=\fontsize{8pt}{10pt}\selectfont}
    \setbeamerfont{block title}{size=\fontsize{8pt}{10pt}\selectfont}
    \setbeamerfont{block body}{size=\fontsize{8pt}{10pt}\selectfont}
    \setbeamerfont{itemize/enumerate body}{size=\fontsize{8pt}{10pt}\selectfont}
    \setbeamerfont{itemize/enumerate subbody}{size=\fontsize{8pt}{10pt}\selectfont}
    \setbeamerfont{itemize/enumerate subsubbody}{size=\fontsize{8pt}{10pt}\selectfont}
    \setbeamerfont{tt text}{size=\fontsize{8pt}{10pt}\selectfont}
    \begin{columns}
        \begin{column}{0.48\textwidth}
            \begin{block}{ 日志存储 (RocksLogStore)}
                \texttt{RocksLogStore} 实现 \texttt{RaftLogStorage<TypeConfig>}:
                \begin{itemize}
                    \item \textbf{持久化}: 使用 RocksDB \texttt{logs} column family 存储日志条目
                    \item \textbf{元数据}: 使用 \texttt{meta} column family 存储 vote 和 last\_purged\_log\_id
                    \item \textbf{方法}:
                          \begin{itemize}
                              \item \texttt{append()}: 追加日志条目并刷新 WAL
                              \item \texttt{truncate()}: 截断冲突日志
                              \item \texttt{purge()}: 清理已应用日志
                              \item \texttt{get\_log\_state()}: 获取日志状态
                          \end{itemize}
                \end{itemize}
            \end{block}
        \end{column}
        \begin{column}{0.48\textwidth}
            \begin{block}{状态机存储 (StateMachineStore)}
                \texttt{StateMachineStore} 实现 \texttt{RaftStateMachine<TypeConfig>}:
                \begin{itemize}
                    \item \textbf{状态管理}: \texttt{BTreeMap<String, String>} 存储 KV 数据
                    \item \textbf{日志应用} (\texttt{apply()}):
                          \begin{itemize}
                              \item 处理 \texttt{Put}/\texttt{Del} 请求,更新状态机
                              \item 处理 \texttt{Membership} 变更
                              \item 更新 \texttt{last\_applied\_log\_id}
                          \end{itemize}
                    \item \textbf{快照}:
                          \begin{itemize}
                              \item \texttt{build\_snapshot()}: 序列化状态机数据
                              \item \texttt{install\_snapshot()}: 从快照恢复状态机
                          \end{itemize}
                \end{itemize}
            \end{block}
        \end{column}
    \end{columns}
    \endgroup
\end{frame}

\begin{frame}{网络层实现}
    \begin{block}{Network 实现 \texttt{RaftNetwork<TypeConfig>}}
        \begin{itemize}
            \item \textbf{RPC 方法}:
                  \begin{itemize}
                      \item \texttt{vote()}: 选举投票
                      \item \texttt{append\_entries()}: 日志复制
                      \item \texttt{install\_snapshot()}: 快照传输
                  \end{itemize}
            \item \textbf{通信方式}: HTTP POST 请求,JSON 序列化
        \end{itemize}
    \end{block}
    \begin{block}{Raft RPC接口}
        \begin{itemize}
            \item \texttt{POST /vote}:选举投票
            \item \texttt{POST /append}:日志复制
            \item \texttt{POST /snapshot}:快照传输
        \end{itemize}
    \end{block}
\end{frame}

\begin{frame}{API层设计 (\texttt{src/network/})}
    \begin{block}{管理API}
        \begin{itemize}
            \item \texttt{POST /init}:初始化集群
            \item \texttt{POST /add-learner}:添加learner节点
            \item \texttt{POST /change-membership}:改变成员关系
            \item \texttt{GET /metrics}:查询集群状态
        \end{itemize}
    \end{block}
    \begin{block}{业务API}
        \begin{itemize}
            \item \texttt{POST /write}:KV写入操作(Put/Del)
            \item \texttt{POST /read}:KV读取操作(单个键值)
            \item \texttt{GET /read-all}:读取所有键值对
        \end{itemize}
    \end{block}
\end{frame}

\begin{frame}[fragile]{写入流程(Put/Delete)}
    \begin{block}{流程说明}
        \texttt{/write} API 接受 \texttt{Request} 类型的请求,支持两种操作:\texttt{Put} 用于插入或更新键值对,\texttt{Del} 用于删除指定键。\texttt{write} 的语义是修改状态机,即添加一条 \texttt{log} 日志。
    \end{block}
    \begin{block}{执行步骤}
        \begin{enumerate}
            \item 客户端调用 \texttt{/write} API
            \item \texttt{Raft.client\_write()} 将请求追加到日志
            \item Leader 复制日志到多数节点
            \item 日志提交后,\texttt{apply()} 应用到状态机
            \item 返回响应给客户端
        \end{enumerate}
    \end{block}
\end{frame}

\begin{frame}[fragile]{写入流程实现}
    \begin{block}{src/mod.rs fn apply() }
        \small
        \begin{verbatim}
EntryPayload::Normal(req) => match req {
    Request::Put { key, value } => {
        let mut st = self.data.kvs.write().await;
        st.insert(key, value.clone());
        Response { value: Some(value) }
    }
    Request::Del { key } => {
        let mut st = self.data.kvs.write().await;
        st.remove(&key);
        Response { value: Some("Ok".to_string()) }
    }
}
\end{verbatim}
    \end{block}
\end{frame}

\begin{frame}{读取流程}
    \begin{block}{流程说明}
        读取操作直接从状态机读取,无需经过 Raft 协议。
    \end{block}
    \begin{block}{执行步骤}
        \begin{enumerate}
            \item 客户端调用 \texttt{/read} 或 \texttt{/read-all} API
            \item 直接从 \texttt{app.key\_values}(状态机数据)读取
            \item 返回查询结果
        \end{enumerate}
    \end{block}
\end{frame}

\begin{frame}[fragile]{读取流程实现}
    \begin{block}{直接从状态机读取 (\texttt{src/network/api.rs})}
        \small
        \begin{verbatim}
#[post("/read")]
pub async fn read(app: Data<App>, req: Json<String>) 
    -> actix_web::Result<impl Responder> {
    let key = req.0;
    let kvs = app.key_values.read().await;
    let value = kvs.get(&key);

    let res: Result<String, Infallible> = 
        Ok(value.cloned().unwrap_or_default());
    Ok(Json(res))
}
        \end{verbatim}
    \end{block}
\end{frame}

\begin{frame}{测试场景}
    \begin{block}{\texttt{test.sh}}
        \fontsize{9.9pt}{11.88pt}\selectfont
        \begin{itemize}
            \item \textbf{集群初始化}:启动 3 个未初始化的节点,初始化节点 1 为单节点集群(Leader)
            \item \textbf{成员管理}:添加节点 2 和节点 3 为 learner,将集群从单节点 \texttt{[1]} 扩展为三节点 \texttt{[1, 2, 3]}
            \item \textbf{数据写入}:在 Leader 节点写入键值对,验证数据已复制到所有节点
            \item \textbf{数据读取}:从所有节点(Leader 和 Followers)读取数据,验证一致性
            \item \textbf{数据删除}:测试删除键值对,验证删除操作在所有节点生效
            \item \textbf{模拟故障}:杀死 Leader 节点(节点 1),验证剩余节点选举出新 Leader,在新 Leader 上继续写入和读取
            \item \textbf{节点恢复}:重启节点 1,验证节点 1 能够从快照或日志恢复状态,验证恢复后数据一致性
        \end{itemize}
    \end{block}
\end{frame}

\section{总结}

\begin{frame}{项目总结}
    \begin{block}{meta-server}
        \begin{itemize}
            \item 实现客户端登录与节点分配
            \item 支持完整数据操作(增删改查)接口
            \item 实现了 raft 节点管理、集群状态监控
        \end{itemize}
    \end{block}
    \begin{block}{kv-store}
        \begin{itemize}
            \item 基于 Openraft 保证据一致性,实现了日志与状态机同步
            \item 使用 RocksDB 进行日志、状态机和数据快照的持久化存储
            \item 测试了节点故障和恢复,具有容错性
        \end{itemize}
    \end{block}
\end{frame}

\backmatter
\end{document}
